\documentclass[a4paper,10pt]{article}
\usepackage[utf8x]{inputenc}

%opening
\title{TTP1 - Protokoll}
\author{Andreas Krohn, Benjamin Vetter}

\begin{document}

\maketitle

\section{Projektschritt 2}
\subsection{Welche Informationen tauschen die Nachbarn bei den jeweiligen Protokollen aus (Neighbor-Nachrichten)? Wie ermitteln/bewerten sie die Qualität der Links?}
\subsubsection*{BATMAN}
Hosts kommunizieren über BATMAN-Pakete. Diese bestehen aus Originator Message (OGM) und optionalen Host Network Announcements (HNA).

OGM enthält:
\begin{enumerate}
  \item Version
  \item Is-direct-link flag
  \item Unidirectional flag
  \item TTL
  \item Gateway flags
  \item Sequence Number
  \item Originator Address
\end{enumerate}
HNA enthält:
\begin{enumerate}
  \item Network Address
  \item Netmask
\end{enumerate}


\subsubsection*{Ermittlung der Qualität und Bidirektionalität der Links - BATMAN}
Siehe RFC Abschnitt 4.2 "`The \emph{amount} of Sequence Numbers recorded in the Sliding Window is used as a metric for the quality of detected links an paths."'

Siehe RFC Abschnitt 5.3 Timeing, vgl. Sequence Number mehrerer OGMs..

\subsubsection*{OLSR}
Es gibt ein generelles Messageformat. Es enthält:
\begin{itemize}
  \item abtippen RFC3626.txt
\end{itemize}

Anhand des Message Type-Feldes wird festgelegt, welcher Nachrichtentyp im MESSAGE-Block enthalten ist. Spezifiziert sind in RFC3626 die Nachrichten HELLO, TC und MID.

HELLO-messages, performing the task of link sensing, neighbor detection and Multipoint Relay (MPR) signaling,



TC-messages, performing the task of topology declaration (advertisement of link states).

MID-messages, performing the task of declaring the presence of multiple interfaces on a node.

\subsubsection*{Ermittlung der Qualität und Bidirektionalität der Links - OLSR}
...RFC lesen..
Link Set über HELLO Messages aufbauen. Eigener Originator in empfangener Message enthalten -> Symmetrischer Link. Timeouts -> Asymmetrisch, dann Tot.
Symmetrie eines Links ist ein Indiz für Qualität.. bla..
Signal Noise Ratio -> l_link_quality P.57f

\subsection{Wie bilden sich die lokalen Mesh-Konfigurationen?}

Proaktiv

\subsection{Ist das B.A.T.M.A.N.-Protokoll wirklich immer Loop-frei? Wie könnten sich ggf. Loops bilden?}
Sind mehrere Nodes sowohl per BATMAN als auch per klassischem Ethernet verbunden, kann eine Loop entstehen. Die BATMAN-spezifischen Paketheader werden im Ethernet nicht weitergeleitet, damit die Mechanismen zur Loopverhinderung ausgehebelt....


\end{document}
