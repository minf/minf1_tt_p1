\documentclass[a4paper,10pt]{article}
\usepackage[utf8x]{inputenc}

\title{TTP1 - Protokoll}
\author{Andreas Krohn, Benjamin Vetter, Benjamin Jochheim}

\begin{document}

\maketitle

\section{Projektschritt 2: Vergleichende Betrachtung des B.A.T.M.A.N.- und OLSR-Protokolls}

\subsection{Welche Informationen tauschen die Nachbarn bei den jeweiligen Protokollen aus (Neighbor-Nachrichten)?}

% B.A.T.M.A.N.

\subsubsection*{B.A.T.M.A.N.}

Bei B.A.T.M.A.N. werden proaktiv sog. Originator-Nachrichten (OGMs) ausgetauscht.
Eine OGM umfasst folgende Felder: 

\begin{description}
  \item[Version] Versionsnummer
  \item[Is-direct-link flag] Kennzeichnet, ob der Knoten ein direkter Nachbar ist oder nicht
  \item[Unidirectional flag] Kennzeichnet, ob der Knoten bidirektional ist oder nicht
  \item[TTL] der Time-To-Live Wert der OGM
  \item[GWFlags] Kennzeichnet ob der Originator als Gateway ins Internet fungiert. Ein Originator kann sich selbst als Internet-Gateway annoncieren. Die Annonce beinhaltet die Link-Geschwindigkeit 
  \item[GWPort] Beinhaltet eine Portnummer für einen Tunnel falls sich der Originator als Internet-Gateway annonciert
  \item[Sequenznummer] Eine Sequenznummer der OGM (0 bis 2\^16-1). Anhand der Sequenznummer können andere Knoten entscheiden ob eine OGM "neu" ist.
  \item[Originator-Adresse] Die IP-Adresse des B.A.T.M.A.N. Interfaces des OGM-Initiators
\end{description}

Einer OGM können sich Host Network Announcements (HNAs) anschließen. 
Eine HNA umfasst folgende Felder: 

\begin{description}
  \item[Netmask] Kennzeichnet die Größe des annoncierten Netzes
  \item[Network address] Die Adresse des annoncierten Netzes
\end{description}

% OLSR

\subsubsection*{OLSR}

Es gibt ein generelles Messageformat. Es enthält:

\begin{itemize}
  \item abtippen RFC3626.txt
\end{itemize}

Anhand des Message Type-Feldes wird festgelegt, welcher Nachrichtentyp im MESSAGE-Block enthalten ist. Spezifiziert sind in RFC3626 die Nachrichten HELLO, TC und MID.
HELLO-messages, performing the task of link sensing, neighbor detection and Multipoint Relay (MPR) signaling,
TC-messages, performing the task of topology declaration (advertisement of link states).
MID-messages, performing the task of declaring the presence of multiple interfaces on a node.

\subsection{Wie ermitteln/bewerten sie die Qualität der Links}

% B.A.T.M.A.N.

\subsubsection*{B.A.T.M.A.N.}

Erhält ein Knoten schneller und zuverlässiger OGMs von einem Nachbarn kann er diesen Knoten als Best-Next-Hop auswählen um Nachrichten an weiter entfernte Knoten zu verschicken.
Hierzu unterhält jeder Knoten ein Sliding-Window bzgl. der Sequenznummern von OGMs jedes Nachbarn. 
Erhält ein Knoten eine OGM von einem Nachbarn betrachtet er die Sequenznummer der OGM.
Wenn die Sequenznummer im Fenster vorhanden ist, speichert der Knoten diese Information.
Hingegen wird das Fenster verschoben, wenn die Sequenznummer außerhalb des Fensters liegt.
Je mehr Sequenznummern innerhalb des Fensters liegen, desto besser bewertet der Knoten die Link-Qualität.
Der Link, der die meisten Sequenznummern in seinem Fenster aufweisen kann, gilt als aktueller Best-Next-Hop.

\subsubsection*{OLSR}

...RFC lesen..
Link Set über HELLO Messages aufbauen. Eigener Originator in empfangener Message enthalten -> Symmetrischer Link. Timeouts -> Asymmetrisch, dann Tot.
Symmetrie eines Links ist ein Indiz für Qualität.. bla..
Signal Noise Ratio -> l\_link\_quality P.57f

\subsection{Wie bilden sich die lokalen Mesh-Konfigurationen?}

% B.A.T.M.A.N.

\subsubsection*{B.A.T.M.A.N.}

B.A.T.M.A.N. ist ein proaktives Routing-Protokoll.
Das Protokoll hat das Ziel für jeden Knoten den jeweils besten Next-Hop zu finden, so dass keine vollständigen Routen verwaltet werden müssen, sondern nur der nächste Hop zum Ziel.
Jeder Knoten broadcastet Originator-Nachrichten (OGMs) in einem Intervall um seine Nachbarn über seine Existenz zu informieren.
Ein Jitter verhindert dabei Kollisionen.
Seine Nachbarn re-broadcasten die OGMs und informieren ihrerseits ihre Nachbarn über die Existenz des eigentlichen Absenders der OGM.
Jeder Knoten re-broadcastet die OGM seines aktuell als Best-Next-Hop identifizierten Knotens, sofern der Knoten die OGM anhand der Sequenznummer als "neu" einstuft und die TTL noch nicht abgelaufen ist.
Beim jedem Re-broadcasting wird der TTL-Wert um 1 verringert.
Daher wird das Netz mit OGMs geflutet.
Eine Verbindung mit einem spezifischen Nachbarn gilt als bidrektional wenn ein Knoten seine eigenen OGMs von diesem Nachbarn empfängt.

% OLSR

\subsubsection*{OLSR}

\subsection{Ist das B.A.T.M.A.N.-Protokoll wirklich immer Loop-frei? Wie könnten sich ggf. Loops bilden?}
Sind mehrere Nodes sowohl per BATMAN als auch per klassischem Ethernet verbunden, kann eine Loop entstehen. Die BATMAN-spezifischen Paketheader werden im Ethernet nicht weitergeleitet, damit die Mechanismen zur Loopverhinderung ausgehebelt....

\end{document}

